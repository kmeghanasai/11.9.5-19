\let\negmedspace\undefined
\let\negthickspace\undefined
\documentclass[journal,12pt,onecolumn]{IEEEtran}
\usepackage{cite}
\usepackage{amsmath,amssymb,amsfonts,amsthm}
\usepackage{algorithmic}
\usepackage{graphicx}
\usepackage{textcomp}
\usepackage{xcolor}
\usepackage{txfonts}
\usepackage{listings}
\usepackage{enumitem}
\usepackage{mathtools}
\usepackage{gensymb}

\usepackage{tkz-euclide} % loads  TikZ and tkz-base
\usepackage{listings}



\newtheorem{theorem}{Theorem}[section]
\newtheorem{problem}{Problem}
\newtheorem{proposition}{Proposition}[section]
\newtheorem{lemma}{Lemma}[section]
\newtheorem{corollary}[theorem]{Corollary}
\newtheorem{example}{Example}[section]
\newtheorem{definition}[problem]{Definition}
%\newtheorem{thm}{Theorem}[section] 
%\newtheorem{defn}[thm]{Definition}
%\newtheorem{algorithm}{Algorithm}[section]
%\newtheorem{cor}{Corollary}
\newcommand{\BEQA}{\begin{eqnarray}}
\newcommand{\EEQA}{\end{eqnarray}}
\newcommand{\system}[1]{\stackrel{#1}{\rightarrow}}

\newcommand{\define}{\stackrel{\triangle}{=}}
\theoremstyle{remark}
\newtheorem{rem}{Remark}
%\bibliographystyle{ieeetr}
\begin{document}
%
\providecommand{\pr}[1]{\ensuremath{\Pr\left(#1\right)}}
\providecommand{\prt}[2]{\ensuremath{p_{#1}^{\left(#2\right)} }}        % own macro for this question
\providecommand{\qfunc}[1]{\ensuremath{Q\left(#1\right)}}
\providecommand{\sbrak}[1]{\ensuremath{{}\left[#1\right]}}
\providecommand{\lsbrak}[1]{\ensuremath{{}\left[#1\right.}}
\providecommand{\rsbrak}[1]{\ensuremath{{}\left.#1\right]}}
\providecommand{\brak}[1]{\ensuremath{\left(#1\right)}}
\providecommand{\lbrak}[1]{\ensuremath{\left(#1\right.}}
\providecommand{\rbrak}[1]{\ensuremath{\left.#1\right)}}
\providecommand{\cbrak}[1]{\ensuremath{\left\{#1\right\}}}
\providecommand{\lcbrak}[1]{\ensuremath{\left\{#1\right.}}
\providecommand{\rcbrak}[1]{\ensuremath{\left.#1\right\}}}
\newcommand{\sgn}{\mathop{\mathrm{sgn}}}
\providecommand{\abs}[1]{\left\vert#1\right\vert}
\providecommand{\res}[1]{\Res\displaylimits_{#1}} 
\providecommand{\norm}[1]{\left\lVert#1\right\rVert}
%\providecommand{\norm}[1]{\lVert#1\rVert}
\providecommand{\mtx}[1]{\mathbf{#1}}
\providecommand{\mean}[1]{E\left[ #1 \right]}
\providecommand{\cond}[2]{#1\middle|#2}
\providecommand{\fourier}{\overset{\mathcal{F}}{ \rightleftharpoons}}
\newenvironment{amatrix}[1]{%
  \left(\begin{array}{@{}*{#1}{c}|c@{}}
}{%
  \end{array}\right)
}
%\providecommand{\hilbert}{\overset{\mathcal{H}}{ \rightleftharpoons}}
%\providecommand{\system}{\overset{\mathcal{H}}{ \longleftrightarrow}}
	%\newcommand{\solution}[2]{\textbf{Solution:}{#1}}
\newcommand{\solution}{\noindent \textbf{Solution: }}
\newcommand{\cosec}{\,\text{cosec}\,}
\providecommand{\dec}[2]{\ensuremath{\overset{#1}{\underset{#2}{\gtrless}}}}
\newcommand{\myvec}[1]{\ensuremath{\begin{pmatrix}#1\end{pmatrix}}}
\newcommand{\mydet}[1]{\ensuremath{\begin{vmatrix}#1\end{vmatrix}}}
\newcommand{\myaugvec}[2]{\ensuremath{\begin{amatrix}{#1}#2\end{amatrix}}}
\providecommand{\rank}{\text{rank}}
\providecommand{\pr}[1]{\ensuremath{\Pr\left(#1\right)}}
\providecommand{\qfunc}[1]{\ensuremath{Q\left(#1\right)}}
	\newcommand*{\permcomb}[4][0mu]{{{}^{#3}\mkern#1#2_{#4}}}
\newcommand*{\perm}[1][-3mu]{\permcomb[#1]{P}}
\newcommand*{\comb}[1][-1mu]{\permcomb[#1]{C}}
\providecommand{\qfunc}[1]{\ensuremath{Q\left(#1\right)}}
\providecommand{\gauss}[2]{\mathcal{N}\ensuremath{\left(#1,#2\right)}}
\providecommand{\diff}[2]{\ensuremath{\frac{d{#1}}{d{#2}}}}
\providecommand{\myceil}[1]{\left \lceil #1 \right \rceil }
\newcommand\figref{Fig.~\ref}
\newcommand\tabref{Table~\ref}
\newcommand{\sinc}{\,\text{sinc}\,}
\newcommand{\rect}{\,\text{rect}\,}
%%
%	%\newcommand{\solution}[2]{\textbf{Solution:}{#1}}
%\newcommand{\solution}{\noindent \textbf{Solution: }}
%\newcommand{\cosec}{\,\text{cosec}\,}
%\numberwithin{equation}{section}
%\numberwithin{equation}{subsection}
%\numberwithin{problem}{section}
%\numberwithin{definition}{section}
%\makeatletter
%\@addtoreset{figure}{problem}
%\makeatother

%\let\StandardTheFigure\thefigure
\let\vec\mathbf

\bibliographystyle{IEEEtran}





\bigskip

\renewcommand{\thefigure}{\theenumi}
\renewcommand{\thetable}{\theenumi}
%\renewcommand{\theequation}{\theenumi}


\title{Discrete Assignment}
\author{Karyampudi Meghana Sai\\ EE23BTECH11031}
\maketitle

\section*{Problem Statement}

The ratio of the A.M and G.M of two positive numbers $a$ and $b$ is $m:n$. Show that $a:b = \brak{ m + \sqrt{m^2 - n^2}} : \brak{ m - \sqrt{m^2 - n^2}}$.
\section*{Solution}

Expressing A.M and G.M in terms of $a$ and $b$:
\begin{align}
\frac{a + b}{2\sqrt{ab}} = \frac{m}{n} \label{eq:11.9.5.19eq1}
\end{align}

Let's assume that $x = \sqrt{\frac{a}{b}}$. Then, we have:
\begin{align}
\frac{a}{b} = x^2 \label{eq:11.9.5.19eq2}
\end{align}

Substituting the value of x in equation \eqref{eq:11.9.5.19eq1}:
\begin{align}
\frac{1 + x^2}{2x} &= \frac{m}{n}\label{eq:11.9.5.19eq3} \\
\frac{1}{x} + x &= \frac{2m}{n} \label{eq:11.9.5.19eq4} \\
x^2 - \frac{2m}{n}x + 1 &=  0 \label{eq:11.9.5.19eq5}\\
\implies x &= \frac{m}{n} \pm \frac{\sqrt{m^2 - n^2}}{n} \label{eq:11.9.5.19eq6}
\end{align}

Since $x = \sqrt{\frac{a}{b}}$, $x$ must be positive.
\begin{align}
x = \frac{m + \sqrt{m^2 - n^2}}{n}\label{eq:11.9.5.19eq7}
\end{align}

Referencing the value of $x$ from equation\eqref{eq:11.9.5.19eq2}.
\begin{align}
\frac{a}{b} &=\brak{\frac{m + \sqrt{m^2 - n^2}}{n}}^2  \label{eq:11.9.5.19eq8}
\end{align}

Multiplying both the numerator and denominator with $\brak{m-\sqrt{m^2 - n^2}}$: 
\begin{align} 
\frac{a}{b} &= \frac{1}{n^2} \frac{\brak{m + \sqrt{m^2 - n^2}}^2  \brak{m-\sqrt{m^2 - n^2}}}{\brak{m-\sqrt{m^2 - n^2}}}\label{eq:11.9.5.19eq9}\\
\implies a:b &= \brak{ m + \sqrt{m^2 - n^2}}: \brak{m - \sqrt{m^2 - n^2}}\label{eq:11.9.5.19eq10}
\end{align}
nth term of the AP :
\begin{align}
y(n)&=\sbrak{a+n\brak{b-a}}u(n)\label{eq:11.9.5.19eq11}\\
n^k u(n) &\system{Z} (-1)^k z^k \frac{d^k}{dz^k}U(z)\label{eq:11.9.5.19eq12}\\
u(n) &\system{Z} \frac{1}{\brak{1 - z^{-1}}} \quad \abs{ z} > \abs{1} \label{eq:11.9.5.19eq13}\\
nu(n) &\system{Z} \frac{z^{-1}}{\brak{1 - z^{-1}}^2} \quad \abs{ z} > \abs{1} \label{eq:11.9.5.19eq14}
\end{align}
Referencing the equations from \eqref{eq:11.9.5.19eq13},\eqref{eq:11.9.5.19eq14}.\\
\begin{align}
y(n) &\system{Z} \frac{a}{\brak{1 - z^{-1}}}+\frac{\brak{b-a}z^{-1}}{\brak{1-z^{-1}}^2} \quad \abs{ z} > \abs{1} \label{eq:11.9.5.19eq15}
\end{align}
nth term of the GP :
\begin{align}
y(n)&=a\brak{{\frac{b}{a}}}^n u(n)\label{eq:11.9.5.19eq16}\\
r^k u(n) &\system{Z} \frac{1}{\brak{1-rz^{-1}}} \quad \abs{ z} > \abs{r} \label{eq:11.9.5.19eq17}
\end{align}
Referencing the equation from \eqref{eq:11.9.5.19eq17}.\\
\begin{align}
y(n) &\system{Z} \frac{a^2 z^{-1}}{\brak{a-bz^{-1}}} \quad \abs{ z} > \abs{\frac{b}{a}}\label{eq:11.9.5.19eq18}
\end{align}
\end{document}

